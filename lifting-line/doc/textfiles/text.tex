\section{Reference}
\href{https://en.wikipedia.org/wiki/Lifting-line_theory}{Lifting-line theory - Wikipedia}

\section{Lifting line theory summary}
This section summarizes the main ideas and equations of lifting line theory:

\subsection{Lift of an airfoil}
The lift produced by an infinitesimal span airfoil $L_y$ in steady state is given by the Kutta–Joukowski theorem: 
\begin{equation}
    \frac{dL}{dy} = L_y = \rho V \Gamma(y, \alpha)
\end{equation}

Where $\Gamma$ represents the circulation around the airfoil. If the airfoil is at an angle $\alpha$, the lift is given by:
\begin{equation}
    \Gamma = V C(y,\alpha) = V 2 \pi c(y) \sin(\alpha(y))
\end{equation}
Where $c(y)$ is the local chord length and $\alpha(y)$ is the local angle of attack considering any wing twist.

The lift can be integrated over the span to obtain the total lift.

\subsection{Lifiting line}
Because the circulation is not constant along the wingspan, the variation in circulation needs to be compensated by the shedding of a vortex. The shed vortices instead induce a vertical velocity over the wingspan.

This induced velocity can be computed by:
\begin{equation}
    w(y) = \int_{-b/2}^{b/2} \frac{d\Gamma(\tilde{y})}{4\pi\left(y - \tilde{y}\right)}
\end{equation}

Note the integrand is singular at $y = \tilde{y}$, so it is understood in terms of the Cauchy principal value.

The circulation at a point can be extended to account for the effect of the induced velocity:
\begin{align}
    \Gamma(y)   &= V 2 \pi c(y)  \left( \alpha(y) + \frac{w(y)}{V}\right) \\
                &= 2 \pi c(y) \left( V \alpha(y) + \int_{-b/2}^{b/2} \frac{d\Gamma(\tilde{y})}{4\pi\left(y - \tilde{y}\right)} \right)
\end{align}

And the total lift and drag can be computed as:
\begin{align}
    L = \rho V  \int_{-b/2}^{b/2} \Gamma(y, \alpha) dy \\
    D = \rho \int_{-b/2}^{b/2} \Gamma(y, \alpha) w(y) dy
\end{align}

Note: wikipedia states the induced drag includes $\alpha(y)$ instead of $w(y)$, however this could lead to induced drag simply due to the angle of attack in the case of an infinite wing.

\section{Solution}
The solution process consists of find the circulation profile for a given wing geometry, i.e. $c(y)$ and $\alpha(y)$.

In order to compute the circulation profile, we can discretize the problem and solve it numerically.

\subsection{Discretization scheme}
Since we are considering an integral in the Cauchy singular value sense, we will discretize the integral using a rectangle rule.

The rectangle rule value is done as follows:
\begin{equation}
    F(x) = \int_{a}^{b} f(x) dx \approx \sum_{i=0}^{N} f(x_i) \Delta_i x_i
\end{equation}

Where $\Delta_i$ is the difference operator which can be expressed as:
\begin{equation}
    \Delta_i f(x_i) = \frac{f(x_{i+1}) - f(x_{i-1})}{2}
\end{equation}

Considering that at the limits we will instead use:
\begin{align}
    \Delta_0 f(x_0) &= \frac{f(x_1) - f(x_0)}{2} \\
    \Delta_N f(x_N) &= \frac{f(x_N) - f(x_{N-1})}{2}
\end{align}

This rule let us compute the Cauchy value by simply excluding the integrand at the singularity point.

\subsubsection{Example of the integration of a Cauchy singular value}
We could check the accuracy of this scheme by integrating:
\begin{equation}
    F(x) = \int_{-1}^{2}\frac{dx}{x}
\end{equation}

The function is odd about the singularity at $x=0$, which is the use case for the Cauchy singular value, so we can split the integration into two parts and play with the limits.
Whose analytic solution is:
\begin{align}
    F(x)    &= \int_{-1}^{-\epsilon} \frac{dx}{x} + \int_{\epsilon}^{2} \frac{dx}{x} = \\
            &= \int_{1}^{\epsilon} \frac{dx}{x} + \int_{\epsilon}^{2} \frac{dx}{x} = \\
            &= \left. \ln(x) \right|_{1}^{\epsilon} + \left. \ln(x) \right|_{\epsilon}^{2} = \\
            &= \ln(2) - \ln(1) = \ln (2)
\end{align}

TODO: check the discretization scheme against this problem.

\subsection{Discretization of the circulation}
Instead of solving the equation directly, we can discretize the problem and solve it numerically:
\begin{equation}
    \Gamma(y_i) = 2 \pi c(y_i) \left( V \alpha(y_i) + \frac{1}{4\pi}\sum_{j=1}^{N} \frac{\Delta_j\Gamma(y_j)}{\left(y_i - y_j\right)} \right)
\end{equation}

Where $\Delta_j$ is the difference operator defined above.

\subsection{Solver}
The will implement an iterative solver using considering the error and a step size $\lambda$, such that:
\begin{equation}
    \tilde{\Gamma}(y_i) = 2 \pi c(y_i) \left( V \alpha(y_i) + \frac{1}{4\pi}\sum_{j=1}^{N} \frac{\Delta_j\Gamma(y_j)}{\left(y_i - y_j\right)} \right)
\end{equation}

And the next value in the iteration will be:
\begin{equation}
    \Gamma(y_i) = \Gamma(y_i) + \lambda \left( \tilde{\Gamma}(y_i) - \Gamma(y_i) \right)
\end{equation}

In the hope that it converges :)